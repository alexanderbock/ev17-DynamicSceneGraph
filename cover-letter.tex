\documentclass{article}

\usepackage[english]{babel}
\usepackage[utf8]{inputenc}
\usepackage{csquotes}
\usepackage{color}

\setlength{\parindent}{0em}
\setlength{\parskip}{0.8em}

\usepackage{titlesec}

\newcommand\stdsection{%
  \titleformat{\section}
    {\normalfont\large\bfseries}{\thesection}{1em}{}
}
\newcommand\largesection{%
  \titleformat{\section}
    {\normalfont\huge\bfseries}{\thesection}{1em}{}
}


\let\oldref\ref
\renewcommand{\ref}[1]{{\color{blue}(\textbf{\oldref{#1}})}}

\begin{document}

\title{Letter of Response for submission ID 291: \\Dynamic Scene Graph: Enabling Scaling, Positioning, and Navigation in the Universe}
\maketitle

Dear Reviewers, \\

We would like to thank the reviewers for the many valuable comments we have received on
our submission.
During the revision, we did our best to address all of the issues brought up and revise the manuscript accordingly.
Below, you will find our detailed responses to the reviews together with accounts of the performed changes.
The remainder of this cover letter is structured as follows:
First, we cite comments from the review, then we summarize the changes in the manuscript to adhere to these comments.
We have inserted references in bold into the review citations to point to the section where the concern is discussed.
If multiple reviewers raised the same concern, the same section is referenced several times.
All references to sections and figures relate to the new layout in the revised submission.

We hope that all your comments have been sufficiently addressed in this revised manuscript. \\\\

Best regards, \\
The Authors \\\\

\newpage

\largesection
\section*{Reviews}
This section contains citations from the reviewers, along with references to our response to your feedback and suggestions.

\stdsection
\section*{Summary Review}

All reviewers agree that the presented technique is to some extent novel and there is a computer graphics use case, which is the field of astronomical visualization.
Also the reviewers agree, that the general style of the writing is good.
However, there are also some substantial weaknesses which lead to a borderline rating of the current version of the submission.
The main points of criticism include that there are no performance considerations given \ref{concern:performance}, the structuring of the manuscript has been questioned as it introduces inconsistencies and confusion due to lack of definitions and explanations directly in section 3 \ref{concern:structure}, as well the comparison or relation to the large field of level of detail methods is missing \ref{concern:lod}.
Also it would be interesting to understand what other potential use cases might benefit from this method \ref{concern:applications}.

As the submission addresses a real-world problem and provides an elegant and effective solution, but also has some weaknesses, we rate it borderline and ask the authors in case of conditional acceptance to look at the detailed review comments and try to improve especially on the mentioned points of criticism.

\section*{Review 4}

The submission "Dynamic Scene Graph: Enabling Scaling, Positioning, and Navigation in the Universe" addresses the challenge of visualizing data exhibiting huge scale differences in distance, size, and resolution.
This often can lead to precision problems, hindering the seamless and simultaneous visualization of very small and very large objects as well as objects at very distant positions.
The authors especially point out the application of visualizing astronomical data.
Thus the topic of the submission is well fitting for this event.

They propose a dynamically assigned frame of reference to provide the highest possible numerical precision for all salient objects in a scene graph. This makes it possible to smoothly navigate and interactively render astronomical fly-byes of planets and stars, and also rendering the detailed geometry of the space ship in the foreground at the same time, for example.

The submission starts with an introduction that is easy to read and follow and which introduces into the challenge of huge data scale differences and all problems associated with that very well. 
It also motivates the work.

After this related work is discussed in section 2.
This section is short but seems sufficient for this work.

Section 3 provides a theoretical background for the sources of floating point inaccuracies.
This section is hard to follow, especially when reading for the first time, and part of this challenge seems to come from the fact, that the authors chose to move major parts of the content to the back of the paper to Appendix A \ref{concern:structure}.
This is problematic and also introduces some inconsistencies.
I would suggest to reconsider or at least fix some of the problematic parts, see below.

Section 4 introduces the Dynamic Scene Graph as well as methods to dynamically update this structure, this is the major contribution of this work.
It is described (and illustrated) in a clear and easy to follow description and one of the major strengths of this submission.

Section 5 describes some selected results from the before mentioned application of astronomical data exploration and visualization.
One additional positive comment can be made about the fact that the authors
have already included the Dynamic Scene Graph into a larger software framework, namely the OpenSpace software framework.
This ensures additional feedback and proof of concept as a larger group of users will become exposed to this new method.

In the end conclusions are drawn and some (limited) thoughts on future extensions are given.
This could be definitely extended a bit, for example I would be very interested to hear from the authors about their thoughts on further challenges in other application fields \ref{concern:future-work}.

As for the problematic structuring of section 3 vs the Appendix A \ref{concern:structure}: 
Appendix A is introducing some definitions, which are used in the remainder of section 3 (as well as to some extent also in the further sections later on, but in front of the Appendix itself). 
For example the definition of the interval arithmetic and symbols is given in the Appendix, but used without any further explanation in most of the equations in section 3.
This is bad and leads to large confusion for the reader.
Also the definition and an example for the catastrophic cancellation is given only in the Appendix, but referred to in the text of Section 3 as well as Section 4.
And further similar examples would be possible.

As the Appendix is introducing major integral parts of the theory and definitions, I would strongly recommend to include it in Section 3, or, if considered impossible, to at least clearly point the reader to it in a more strict way, mentioning that it is an integral part and necessary to follow and understand the rest of the paper.
Just pointing the "interested" reader to this Appendix seems not sufficient enough.

Other errors \ref{concern:typos} include:

\begin{enumerate}
\item Section 4.1, paragraph 2: $v_Ji$ is the transformation from node i to j and $v_j,i=-v_i,j$ as the inverse.
Here $v_{ji}$ is wrongly displayed as $v_Ji$
\item Section 4.1, paragraph 3: By utilizing the shortest path between M and N we avoid large the (should be: the large) translations which would otherwise be necessary if the all (should be: all the) transformations originate ...
\item Section 4.2., end of paragraph 1: "Likewise, camera B is inside nodes 1,2 and 5 and ...
" should be "Likewise, camera B is inside nodes 1,3 and 5 ..."" if I understand the Figure 4 correctly.
\item Section 5, paragraph about "Space craft.": "As described in Section ??" -- here a wrong reference is apparent.
\item Appendix A, last paragraph: there seems to be something wrong with the statement that "we note that $e_{ijk}=0$ if $A_{ik}=0$ or $A_{ik}=0$" -- here $A_{ik}$ is mentioned twice.
\end{enumerate}

\section*{Review 1}

 The authors present the technique dynamic scene graph to visualize interstellar objects.
The paper is structured and written well.
Only at page 7 there is a missing reference \ref{concern:typos}.
The introduction gives a nice insight into the very different magnitudes of space objects and their relations as well as the challenges during rendering.
In the following, authors discuss related work with respect to navigation interaction, space and education, as well as some solutions to the cancellation effects of floating point numbers.
However, they completely omit the whole world of level-of-detail algorithms out there in computer graphics for decades \ref{concern:lod}.
Authors should do an extensive literature research in this direction, since many proposed solutions were already solved in different ways several years ago.
This is also true for numerical problems of large-scale astronomical simulations; which astrophysics also have dealt with for decades now.

The third section motivates the presented approach.
Authors examine how different operations of vectors and matrices, which are typically applied during rendering, effect precision.
In Equation (1) there is the variable $c_w$ and one circular operator which are not explained \ref{concern:structure}. 
The analysis shows that mainly translations result in problematic cancellations and the authors show the well-known effect that non-linear depth buffers result in precision problems.
The following section describes the algorithms of the dynamic scene graph.
Authors propose a navigation that lets the camera travel in space by traversing the scene graph and successively choosing respective points of interest.
During the traversals, authors try to reduce accuracy problems by using direct traversal in the hierarchy and utilizing respective matrices.
This procedure can also be adapted to movements of cameras or even objects.
The depth-buffer problems are removed by the well-known approach of linearization, using actual distance.

In the results section, authors show many nice rendering of interstellar objects.
However, authors do not present at which hardware these renderings were performed at which performance \ref{concern:performance}.
Also, they present a volumetric rendering of the Milky Way where it remains completely unclear how the volumetric rendering was performed and how this was integrated with the proposed rendering pipeline \ref{concern:volumetric}.
In addition, comparisons to the state of the art or alternative approaches are completely missing \ref{concern:comparisons}.

\section*{Review 2}

In this paper, authors propose a dynamic scene graph.
This method enables fast and accurate scaling, positioning, and navigation without significant loss of precision.
The method for rendering objects with a wide depth range and precision without the explicit need for near and far planes.
    
The dynamic scene graph method is quite novel.
It is an accurate, effective and seamless positioning and navigation by using a relative scene traversal performed and a dynamic camera.
By this way, it can navigate a scene encompassing scales and distances much larger than floating point and precision would normally permit.
However, the author needs to give the performance configuration of the experiment and give a performance comparison with the traditional methods \ref{concern:performance}. 
In order to prove that the framework can solve the problems.
    
The evaluation part of this paper should also be improved \ref{concern:evaluation}.
It should be written more concretely.
The experiment setups, the dataset used should be stated clearly because it may largely affect the results and the patterns of the study.

\section*{Review 3}

This paper addresses the precision problem which is common in the fields of graphics and visualization.
After analyzing the reasons behind the precision problem, this paper proposes dynamic scene graph-based method for seamlessly high-precision rendering of large scale scenes.
The case studies as well as the supplemental video demonstrate good results by using the proposed method.
The proposed technique is reproducible for a skilled graduate student.


\largesection
\section*{Improvements}
In this section, we respond to your comments and describe how we were able to improve the manuscript based on the feedback.

%Suggested improvements
%The primary reviewer points out the following possible improvements

%Adding performance considerations.
%Moving definitions from the appendix into section three.
%Describing how our method relates to level of detail methods.
%Adding some more use cases that may benefit from our method.

%Additionally, reviewers have also suggested the following:

%5.
%Improving the manuscript by further describing how the volumetric rendering displayed in figure 10 was integrated with our rendering %pipeline.
%6.
%Improving the comparison between our approach to the state of the art and alternative approaches.

%Furthermore, one reviewer noticed a missing reference and a few typos.
%Thanks to the feedback, we have been able to correct those.
%Below we provide a discussion of how we have improved the manuscript based on the reviewer’s suggestions listed above.

%Here we respond to the review statements from the summary review:

\stdsection
\section{Performance considerations}\label{concern:performance}
We added discussions about the performance implications of our method in multiple places
throughout the manuscript;  the end of section 4, section 4.2, and the results section.

Section 4 has been improved to motivate why the proposed scene graph traversal scheme does not increase the algorithmic complexity compared to a standard scene graph traversal.
Furthermore, we have now clarified the algorithmic complexity of updating the camera attachment node.
Section 5 was also modified to clarify that the implementation of the Dynamic Scene Graph in OpenSpace has not impacted performance negatively, compared to the traditional scene graph traversal that was previously implemented.

\section{Theory section and appendix}\label{concern:structure}
We have carefully revised the paper based on reviewers feedback about the structuring of section 3 and the appendix A.
The contents of the appendix have now been incorporated in the theory section to improve the flow of the paper.
In the revised version, we thorougly introduce notation and concepts of interval arithmetic before analyzing the rendering pipeline.

\section{Relation to level of detail methods}\label{concern:lod}
We added references to section in order to related and delineate our work from the 
previous work in the field of level of detail and other multiresolution methods.
Furthermore, we have clarified how our method differs from existing work in level of detail.

\section{Other applications}\label{concern:applications}
In the introduction we have elaborated on the possible use cases of our method.

\section{Future work}\label{concern:future-work}
In order to address this valid concern, we included additional explanations in the
introduction, the beginning of the results sections, as well as providing avenues for
future work. Furthermore, we included more potential avenues for future research in the
end of the manuscript, especially dealing with the integration of datasets that span large
extents both spatially and temporally, which is possible due to our proposed method.

\section{Typographical corrections and clarifications}\label{concern:typos}
We addressed these comments and typographical errors where they occurred in the
manuscript.

\section{Comparison to other methods}\label{concern:comparisons}
TODO.

\section{Rendering volumetric data}\label{concern:volumetric}
TODO.

\section{Evaluation}\label{concern:evaluation}
TODO.


\end{document}