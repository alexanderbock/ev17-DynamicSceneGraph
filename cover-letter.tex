\documentclass{article}

\usepackage[english]{babel}
\usepackage[utf8]{inputenc}
\usepackage{csquotes}
\usepackage{xcolor}
\usepackage{hyperref}
\usepackage{amsmath}
\usepackage{fullpage}
\usepackage{tcolorbox}

\widowpenalty10000
\clubpenalty10000

\usepackage[margin=1.75in]{geometry}

\hypersetup{
    colorlinks = true,
    linkbordercolor = {white}
}

\setlength{\parindent}{0em}
\setlength{\parskip}{0.8em}

\date{}
 
\usepackage{titlesec}


%\let\oldref\ref
%\renewcommand{\ref}[1]{{\color{blue}(\textbf{\oldref{#1}})}}

\begin{document}

\title{Letter of Response for submission ID 291: \\Dynamic Scene Graph: Enabling Scaling, Positioning, and Navigation in the Universe}
\maketitle

Dear Reviewers, \\

We would like to thank you for the many valuable comments we received for our submission.
During the revision, we addressed the issues brought up and revised the manuscript accordingly.
In order to accommodate the necessary changes, the tenth page now contains both the manuscript as well as the reference, in accordance with the submission guideline checklist.
Below, you will find our responses to the reviews together with descriptions of the performed changes.
The remainder of this cover letter is structured as follows:

For each concern that was raised by a reviewer, we paraphrase the reviewer's comments and then summarize the changes that we made in the manuscript to address the concern.
Finally, we have inserted the original review texts along with references to the sections where we have discussed the particular concerns.

All references to sections and figures relate to the new layout of the revised submission.

%We hope that all your comments have been sufficiently addressed in this revised manuscript. \\\\

Best regards, \\
The Authors \\\\

\newpage

\section{Reviews and comments}
For each concern that was raised by a reviewer, we paraphrase the reviewer's comments and then summarize the changes that we made in the manuscript to address the concern.

\vspace*{1cm}

\begin{tcolorbox}
\subsection{Clarity in the Theory Section}\label{concern:structure}
Several reviewers agreed that there were ``inconsistencies and confusion due to lack of definitions and explanations directly in Section 3'' and that the ``section is hard to follow, especially when reading for the first time'', which ``come from the fact, that the authors chose to move major parts of the content to the back of the paper to Appendix A''.
\end{tcolorbox}
We carefully revised the paper based on reviewers' feedback about the structuring of Section 3 and the Appendix A.
The contents of the appendix were incorporated into Section 3 in order to make it self-contained and thus improve the flow of the paper.
In the revised version, we introduce the notation and the concepts of interval arithmetic prior to our analysis of the rendering pipeline thus making it easier to follow.
Furthermore, we have refined our notation to highlight the distinction between interval matrices and scalar matrices.

\vspace*{1cm}


\begin{tcolorbox}
\subsection{Comparison to prior work}\label{concern:comparisons}
Reviewers raised the point that ``a comparison or relation to the large field of level of detail methods is missing'' and ``comparisons to the state of the art or alternative approaches are completely missing''.
\end{tcolorbox}
To address this point, we improved the related works section and added references to Section 2 in order to delineate the relation of our work to the previous work in the field of level of detail and other multiresolution methods.

The results section has been improved by clarifying that Figure 3 shows the result of our method compared to a method using power scaled coordinates, which is one of the published state of the art methods for large scale astronomical visualization.
As the only published paper about the ScaleGraph is the 2010 EuroGraphics Area Paper [KHE*10] and this paper does not contain enough technical details to reproduce the method, we have not been able to provide a scientific comparison with that method.
However, as pointed out in the related work section, the Uniview software which is based on the ScaleGraph does not properly handle stereoscopic rendering when transitioning between scenes.
In Section 4.4, we describe how our method handles this problem.

\vspace*{1cm}

\begin{tcolorbox}
\subsection{Other use cases}\label{concern:applications}
These points questioned which ``other potential use cases might benefit from this method'' and that ``(limited) thoughts on future extensions are given'' and the reviewer noted that he ``would be very interested to hear from the authors about their thoughts on further challenges in other application fields''.
\end{tcolorbox}
In order to address this valid concern, we included additional explanations in the introduction, the beginning of the results sections, as well as providing avenues for future work in the end of the manuscript, especially dealing with the integration of datasets that span large extents both spatially and temporally, which is possible due to our proposed method.

\vspace*{1cm}

\begin{tcolorbox}
\subsection{Performance considerations}\label{concern:performance}
The reviewer raises the concern that ``no performance considerations [are] given'' for our proposed method.
\end{tcolorbox}
We added discussions about the performance implications of our method in multiple places throughout the manuscript; the end of Section 4, Section 4.2, and the results section now discuss this issue.

Section 4 has been improved to show why our proposed scene graph traversal scheme does not increase the algorithmic complexity compared to a standard scene graph traversal and might, dependening on the scene graph structure, even be more performant as only a local subsection of the sceen graph needs to be evaluated.

Furthermore, we added a description about the algorithmic complexity of updating the camera attachment node to show that this operation also does not introduce a performace degredation.
Section 5 was modified to clarify that the implementation of the Dynamic Scene Graph in our reference implementation OpenSpace has not impacted performance negatively, compared to the Power Scaled Coordinate method that was previously implemented in the software.
Due to the dependence on a scene graph layout (and thus general irreproducibility), the fact that we do not claim to have an algorithm that improves the performance, and a restriction in the number of pages in the manuscript, we chose to omit a detailed discussion of performance measurements in a separate section of the manuscript.

\vspace*{1cm}

\begin{tcolorbox}
\subsection{Volume rendering}\label{concern:volumetric}
Reviewer 1 raises the issue that we ``present a volumetric rendering of the Milky Way where it remains completely unclear how the volumetric rendering was performed and how this was integrated with the proposed rendering pipeline''.
\end{tcolorbox}
We added a description to the end of Section 5, describing in more detail how volumetric rendering is integrated with the Dynamic Scene Graph in the OpenSpace implementation in particular and how it can be integrated in a generic Dynamic Scene Graph implementation.

\vspace*{1cm}

\begin{tcolorbox}
\subsection{Typographical corrections}\label{concern:typos}
Reviewer 1 and 4 noticed a missing reference and some typographical errors.
\end{tcolorbox}
Thanks to the feedback, these concerns have been corrected in the new version of the manuscript.



%We would like to thank the reviewers for the many valuable comments we have received on
%our submission. During the revision, we did our best to address all of the issues brought
%up and revise the manuscript accordingly. Below, you will find our detailed responses to
%the reviews together with accounts of the performed changes. The remainder of this cover
%letter is structured as follows: First, we cite comments from the review, then we
%summarize the changes in the manuscript to adhere to these comments. All replies are
%structured based on the specific reviewers. If multiple reviewers raised the same concern,
%this point is only discussed at the first appearance and omitted later. All references to
%sections and figures relate to the new layout in the revised submission.
%
%We hope that all your comments have been sufficiently addressed in this revised manuscript.
%
%
\section{Original review texts}
This section contains citations from the reviews, along with references to the previous sections where the raised concerns have been discussed. For a compact presentation, we have omitted paragraphs with solely positive or neutral feedback.
%
%
\subsection*{Summary Review}

%All reviewers agree that the presented technique is to some extent novel and there is a computer graphics use case, which is the field of astronomical visualization.
%Also the reviewers agree, that the general style of the writing is good.
%However, there are also some substantial weaknesses which lead to a borderline rating of the current version of the submission.
[...]

The main points of criticism include that there are no performance considerations given \eqref{concern:performance}, the structuring of the manuscript has been questioned as it introduces inconsistencies and confusion due to lack of definitions and explanations directly in section 3 \eqref{concern:structure}, as well the comparison or relation to the large field of level of detail methods is missing \eqref{concern:comparisons}.
Also it would be interesting to understand what other potential use cases might benefit from this method \eqref{concern:applications}.

[...]

%As the submission addresses a real-world problem and provides an elegant and effective solution, but also has some weaknesses, we rate it borderline and ask the authors in case of conditional acceptance to look at the detailed review comments and try to improve especially on the mentioned points of criticism.

\subsection*{Review 4}

%The submission "Dynamic Scene Graph: Enabling Scaling, Positioning, and Navigation in the Universe" addresses the challenge of visualizing data exhibiting huge scale differences in distance, size, and resolution.
%This often can lead to precision problems, hindering the seamless and simultaneous visualization of very small and very large objects as well as objects at very distant positions.
%The authors especially point out the application of visualizing astronomical data.
%Thus the topic of the submission is well fitting for this event.

%They propose a dynamically assigned frame of reference to provide the highest possible numerical precision for all salient objects in a scene graph. This makes it possible to smoothly navigate and interactively render astronomical fly-byes of planets and stars, and also rendering the detailed geometry of the space ship in the foreground at the same time, for example.

%The submission starts with an introduction that is easy to read and follow and which introduces into the challenge of huge data scale differences and all problems associated with that very well. 
%It also motivates the work.

%After this related work is discussed in section 2.
%This section is short but seems sufficient for this work.

[...]

Section 3 provides a theoretical background for the sources of floating point inaccuracies.
This section is hard to follow, especially when reading for the first time, and part of this challenge seems to come from the fact, that the authors chose to move major parts of the content to the back of the paper to Appendix A \eqref{concern:structure}.
This is problematic and also introduces some inconsistencies.
I would suggest to reconsider or at least fix some of the problematic parts, see below.

%Section 4 introduces the Dynamic Scene Graph as well as methods to dynamically update this structure, this is the major contribution of this work.
%It is described (and illustrated) in a clear and easy to follow description and one of the major strengths of this submission.

%Section 5 describes some selected results from the before mentioned application of astronomical data exploration and visualization.
%One additional positive comment can be made about the fact that the authors have already included the Dynamic Scene Graph into a larger software framework, namely the OpenSpace software framework.
%This ensures additional feedback and proof of concept as a larger group of users will become exposed to this new method.

[...]

In the end conclusions are drawn and some (limited) thoughts on future extensions are given.
This could be definitely extended a bit, for example I would be very interested to hear from the authors about their thoughts on further challenges in other application fields \eqref{concern:applications}.

As for the problematic structuring of section 3 vs the Appendix A \eqref{concern:structure}: 
Appendix A is introducing some definitions, which are used in the remainder of section 3 (as well as to some extent also in the further sections later on, but in front of the Appendix itself). 
For example the definition of the interval arithmetic and symbols is given in the Appendix, but used without any further explanation in most of the equations in section 3.
This is bad and leads to large confusion for the reader.
Also the definition and an example for the catastrophic cancellation is given only in the Appendix, but referred to in the text of Section 3 as well as Section 4.
And further similar examples would be possible.

As the Appendix is introducing major integral parts of the theory and definitions, I would strongly recommend to include it in Section 3, or, if considered impossible, to at least clearly point the reader to it in a more strict way, mentioning that it is an integral part and necessary to follow and understand the rest of the paper.
Just pointing the "interested" reader to this Appendix seems not sufficient enough.

Other errors \eqref{concern:typos} include:

\begin{enumerate}
\item Section 4.1, paragraph 2: $v_Ji$ is the transformation from node i to j and $v_j,i=-v_i,j$ as the inverse.
Here $v_{ji}$ is wrongly displayed as $v_Ji$
\item Section 4.1, paragraph 3: By utilizing the shortest path between M and N we avoid large the (should be: the large) translations which would otherwise be necessary if the all (should be: all the) transformations originate ...
\item Section 4.2., end of paragraph 1: "Likewise, camera B is inside nodes 1,2 and 5 and ...
" should be "Likewise, camera B is inside nodes 1,3 and 5 ..."" if I understand the Figure 4 correctly.
\item Section 5, paragraph about "Space craft.": "As described in Section ??" -- here a wrong reference is apparent.
\item Appendix A, last paragraph: there seems to be something wrong with the statement that "we note that $e_{ijk}=0$ if $A_{ik}=0$ or $A_{ik}=0$" -- here $A_{ik}$ is mentioned twice.
\end{enumerate}

\subsection*{Review 1}

The authors present the technique dynamic scene graph to visualize interstellar objects.
The paper is structured and written well.
Only at page 7 there is a missing reference \eqref{concern:typos}.
The introduction gives a nice insight into the very different magnitudes of space objects and their relations as well as the challenges during rendering.
In the following, authors discuss related work with respect to navigation interaction, space and education, as well as some solutions to the cancellation effects of floating point numbers.
However, they completely omit the whole world of level-of-detail algorithms out there in computer graphics for decades.
Authors should do an extensive literature research in this direction, since many proposed solutions were already solved in different ways several years ago \eqref{concern:comparisons}. 
This is also true for numerical problems of large-scale astronomical simulations; which astrophysics also have dealt with for decades now. 

The third section motivates the presented approach.
Authors examine how different operations of vectors and matrices, which are typically applied during rendering, effect precision.
In Equation (1) there is the variable $c_w$ and one circular operator which are not explained \eqref{concern:structure}. 
The analysis shows that mainly translations result in problematic cancellations and the authors show the well-known effect that non-linear depth buffers result in precision problems.
The following section describes the algorithms of the dynamic scene graph.
Authors propose a navigation that lets the camera travel in space by traversing the scene graph and successively choosing respective points of interest.
During the traversals, authors try to reduce accuracy problems by using direct traversal in the hierarchy and utilizing respective matrices.
This procedure can also be adapted to movements of cameras or even objects.
The depth-buffer problems are removed by the well-known approach of linearization, using actual distance.

In the results section, authors show many nice rendering of interstellar objects.
However, authors do not present at which hardware these renderings were performed at which performance \eqref{concern:performance}.
Also, they present a volumetric rendering of the Milky Way where it remains completely unclear how the volumetric rendering was performed and how this was integrated with the proposed rendering pipeline \eqref{concern:volumetric}.
In addition, comparisons to the state of the art or alternative approaches are completely missing \eqref{concern:comparisons}.

\subsection*{Review 2}

%In this paper, authors propose a dynamic scene graph.
%This method enables fast and accurate scaling, positioning, and navigation without significant loss of precision.
%The method for rendering objects with a wide depth range and precision without the explicit need for near and far planes.
    
%The dynamic scene graph method is quite novel.
%It is an accurate, effective and seamless positioning and navigation by using a relative scene traversal performed and a dynamic camera.
%By this way, it can navigate a scene encompassing scales and distances much larger than floating point and precision would normally permit.

[...]

However, the author needs to give the performance configuration of the experiment and give a performance comparison with the traditional methods \eqref{concern:performance}. 
In order to prove that the framework can solve the problems.
    
The evaluation part of this paper should also be improved \eqref{concern:comparisons} \eqref{concern:performance}.
It should be written more concretely.
The experiment setups, the dataset used should be stated clearly because it may largely affect the results and the patterns of the study.

\subsection*{Review 3}

This paper addresses the precision problem which is common in the fields of graphics and visualization.
After analyzing the reasons behind the precision problem, this paper proposes dynamic scene graph-based method for seamlessly high-precision rendering of large scale scenes.
The case studies as well as the supplemental video demonstrate good results by using the proposed method.
The proposed technique is reproducible for a skilled graduate student.


%%Suggested improvements
%%The primary reviewer points out the following possible improvements
%
%%Adding performance considerations.
%%Moving definitions from the appendix into section three.
%%Describing how our method relates to level of detail methods.
%%Adding some more use cases that may benefit from our method.
%
%%Additionally, reviewers have also suggested the following:
%
%%5.
%%Improving the manuscript by further describing how the volumetric rendering displayed in figure 10 was integrated with our rendering %pipeline.
%%6.
%%Improving the comparison between our approach to the state of the art and alternative approaches.
%
%%Furthermore, one reviewer noticed a missing reference and a few typos.
%%Thanks to the feedback, we have been able to correct those.
%%Below we provide a discussion of how we have improved the manuscript based on the reviewer’s suggestions listed above.
%
%%Here we respond to the review statements from the summary review:
%
%
%\subsection*{Performance considerations}\label{concern:performance}
%
%
%\subsection*{Theory section and appendix}\label{concern:structure}
%
%\subsection{Relation to level of detail methods}\label{concern:lod}
%
%\subsection{Other applications}\label{concern:applications}
%We have elaborated on the possible use cases of our method in the introduction. 
%
%
%\subsection{Typographical corrections and clarifications}\label{concern:typos}
%We addressed the comments and typographical errors that were pointed out by the reviewers where they occurred in the manuscript.
%
%\subsection{Comparison to other methods}\label{concern:comparisons}
%
%\subsection{Rendering volumetric data}\label{concern:volumetric}
%We have added a description to the end of section 5, describing in more detail how volumetric rendering is integrated with the Dynamic Scene Graph in the OpenSpace implementation.

\end{document}
