% ---------------------------------------------------------------------------
% Author guideline and sample document for EG publication using LaTeX2e input
% D.Fellner, v1.13, Jul 31, 2008

\documentclass{egpubl}
\usepackage{eurovis2017}

% --- for  Annual CONFERENCE
% \ConferenceSubmission   % uncomment for Conference submission
\ConferencePaper        % uncomment for (final) Conference Paper
% \STAR                   % uncomment for STAR contribution
% \Tutorial               % uncomment for Tutorial contribution
% \ShortPresentation      % uncomment for (final) Short Conference Presentation
% \Areas                  % uncomment for Areas contribution
% \MedicalPrize           % uncomment for Medical Prize contribution
% \Education              % uncomment for Education contribution
%
% --- for  CGF Journal
% \JournalSubmission    % uncomment for submission to Computer Graphics Forum
% \JournalPaper         % uncomment for final version of Journal Paper
%
% --- for  CGF Journal: special issue
% \SpecialIssueSubmission    % uncomment for submission to Computer Graphics Forum, special issue
\SpecialIssuePaper         % uncomment for final version of Journal Paper, special issue
%
% --- for  EG Workshop Proceedings
% \WsSubmission    % uncomment for submission to EG Workshop
% \WsPaper         % uncomment for final version of EG Workshop contribution
%
 \electronicVersion % can be used both for the printed and electronic version

% !! *please* don't change anything above
% !! unless you REALLY know what you are doing
% ------------------------------------------------------------------------

% for including postscript figures
% mind: package option 'draft' will replace PS figure by a filname within a frame
\ifpdf \usepackage[pdftex]{graphicx} \pdfcompresslevel=9
\else \usepackage[dvips]{graphicx} \fi

\PrintedOrElectronic

% prepare for electronic version of your document
\usepackage{t1enc,dfadobe}

\usepackage{egweblnk}
\usepackage{cite}
\usepackage[utf8]{inputenc}

\usepackage{amsmath}
\usepackage{overlay}
\usepackage{mleftright}
\usepackage{dsfont}
\usepackage{tikz}

\newcommand\constructosum[3]{%
    \begin{tikzpicture}[baseline=(char.base), inner sep=0, outer sep=0]
        \draw (#1,0) circle (#2); 
        \node (char) at (0,0) {$#3\sum$}; % Want to define a second symbol for inline...
    \end{tikzpicture}%
}

\newcommand{\osum}{\mathop{\mathchoice
        {\constructosum{-0.3ex}{0.1}{\displaystyle}}
        {\constructosum{-0.3ex}{0.06}{\textstyle}}
        {\constructosum{-0.2ex}{0.04}{\scriptstyle}}
        {\constructosum{-0.15ex}{0.03}{\scriptscriptstyle}}
    }\displaylimits
}

\newcommand{\joncomment}[1]{\textbf{[-Jonathas-~}
		\textcolor{orange}{#1}
		%\marginpar{\textcolor{blue}{\centerline{{\Huge \textbf{!}}}}}
		\textbf{~]}}

\newcommand{\emilcomment}[1]{\textbf{[-Emil-~}
		\textcolor{red}{#1}
		%\marginpar{\textcolor{blue}{\centerline{{\Huge \textbf{!}}}}}
		\textbf{~]}}

\newcommand{\alexcomment}[1]{\textbf{[-Alex-~}
		\textcolor{blue}{#1}
		%\marginpar{\textcolor{blue}{\centerline{{\Huge \textbf{!}}}}}
		\textbf{~]}}

%\newcommand{\osum}[2]
 %{\overlay{\circ}{\sum_{#1}^{#2}}}


% For backwards compatibility to old LaTeX type font selection.
% Uncomment if your document adheres to LaTeX2e recommendations.
% \let\rm=\rmfamily    \let\sf=\sffamily    \let\tt=\ttfamily
% \let\it=\itshape     \let\sl=\slshape     \let\sc=\scshape
% \let\bf=\bfseries

% end of prologue

\input{dsg-paper-body.inc}
